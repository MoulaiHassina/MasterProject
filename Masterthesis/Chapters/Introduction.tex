% Chapter 0

\chapter*{Introduction générale} % Main chapter title
\addcontentsline{toc}{chapter}{Introduction}

\label{Introduction} % For referencing the chapter elsewhere, use \ref{Introduction} 
% Chapter 0


Une vague d'avancées technologiques a submergé le monde durant cette dernière décennie, l'intelligence artificielle est au centre de toutes les innovations étant par conséquent le sujet le plus actuel et un critère d'évaluation du développement des pays du monde.


Cet intérêt croissant a permis à des villes intelligentes de voir le jour ainsi que des voitures autonomes ou encore des robots sociaux comme Sophia \cite{sofia}.\\

Le développement connu par l'intelligence artificielle peut être expliqué par des besoins accrus et nécessaires de gain de temps, d'efforts et de mains d'œuvres humaines dans l'accomplissement de tâches quotidiennes.

Dans le but de participer à ce développement, les chercheurs ont essayé de proposer des solutions automatisant certaines tâches quotidiennes coûteuses en temps ou jugées dangereuses pour l'homme.
Les systèmes multi-robots font partie des innovations émergentes et solutions proposées pour répondre à ces besoins. Leur usage s'étale sur des domaines d'application des plus intéressants, allant de l'usage domestique au domaine militaire, passant par l'industrie ou encore le domaine de la santé. Ces multiples applications exploitent l'adaptabilité des robots qui sont capables de résoudre divers problèmes \cite{uses}.\\


Notons que la mobilité des robots est l'une de leurs capacités la plus convoitée, que ce soit dans l'assemblage et l'entreposage, le transport industriel, les équipements de nettoyage du sol  \cite{novel}, l'exploration de planète, la recherche et le secourisme des victimes dans une zone sinistrée ou encore la localisation de mines ou de bombes \cite{Dadgar2016}.

De plus, les systèmes multi-robots se caractérisent par leur coopération, celle-ci permet d'établir une meilleure planification des actions et comportements des robots face au monde qui les entoure pour aboutir à un but commun. 
\\
% Cette coopération  a pour but de simuler interactions et processus de raisonnement humain. D’ailleurs une équipe de robots, bien qu'ayant un comportement simple ou n'étant pas puissant individuellement peut compenser et surpasser ses limites grâce à la coopération \cite{Anderson2008}.\\

Ces systèmes multi-robots ont été particulièrement efficaces lorsqu'il s'agissait de problèmes de recherche ou détection, que ce soit dans des espaces connus ou inconnus. L'objectif consiste à localiser une ou plusieurs cibles se trouvant dans un certain espace qu'on appelle environnement. Une cible au sens large peut représenter des personnes perdues, des mines, de l'or, des puits de pétrole et bien plus encore.\\

La détection de cibles est un problème  complexe dont il découle un certain nombre de problématiques. On cite, l'optimisation des temps et efforts des robots à travers un choix adéquat de structure de coopération.  La conception d'un système d'évitement d'obstacles et de navigation fiable est primordial pour garantir des déplacements sans collisions au sein d'environnements inconnus, pouvant contenir des obstacles. Mais encore, les robots doivent adopter des stratégies de recherche intelligentes en raison de leur manque d'information sur l'environnement.\\
 


À travers ce travail, nous chercherons à proposer de nouvelles méthodes de résolution pour la détection de cibles en tentant d'apporter des améliorations au niveau des performances, c'est à dire qu'on vise à minimiser l'effectif (le moins de robots possibles) pour couvrir toute la zone de recherche, cela en trouvant le maximum de cibles.\\

 Notre contribution comportera une adaptation d'un système multi-robots aux quatre approches inspirées de l'intelligence en essaim, à savoir \textit{BSO} et \textit{Multi-BSO} qui sont basées sur le comportement des abeilles lors de la recherche du pollen, puis \textit{EHO} et \textit{ESWSA} qui sont inspirées de l'intelligence des éléphants en termes de communication et planification lors de la recherche d'eau et de nourriture. Un robot se comportera comme une abeille, un essaim d'abeilles ou un éléphant selon l'approche choisie.

Ces quatre approches devront être adaptées à une même modélisation du problème afin de permettre une comparaison effective des résultats obtenus. Pour cette même raison, la stratégie d'évitement d'obstacles sera exploitée et intégrée pour toutes nos méta-heuristiques.\\ 

Ce mémoire compte quatre chapitres organisés comme suit :

D'abord, une synthèse sur les maintes variantes du problème et tous les paramètres influant la direction qu'il peut prendre est présentée dans le chapitre 1.
Les algorithmes d'intelligence en essaim choisis pour l'élaboration de nos solutions sont décrits dans le chapitre 2.
Le chapitre 3 fera l'objet de la modélisation de notre problème et des stratégies d'évitement d'obstacles.
Le chapitre 4 comportera les deux premières contributions basées intelligence des essaims d'abeilles à savoir BSO et Multi-BSO. Deux autres contributions concernant deux algorithmes s'inspirant des éléphants qui sont EHO et ESWSA, seront présentées dans le chapitre 5.

L'efficacité des approches proposées sera validée à travers plusieurs expérimentations dont les résultats seront exhibés au chapitre 6. 
Enfin nous terminerons par une conclusion sur l'ensemble du travail effectué ainsi que des perspectives.









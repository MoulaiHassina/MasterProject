\let\cleardoublepage\clearpage
%\selectlanguage{french}
\begin{poliabstract}{Résumé}

La frontière entre la fiction et la réalité devient chaque jour plus mince, due aux évolutions technologiques qui ont émergé au fil des années. Les robots ne font plus partie de la science-fiction seulement, ils se sont installés dans notre quotidien offrant de nouvelles capacités à l’homme pour sa vie de tous les jours.
Les systèmes multi-robots ont conquis le monde de la recherche dans multiples domaines grâce à l’intelligence des robots qui est due à leur coopération lors de la planification des tâches. C’est la raison pour laquelle plusieurs études se sont donné comme but de les rendre de plus en plus performants, robustes et efficaces.

L’un de leurs domaines d’application les plus récemment exploitées est la recherche de cibles. C’est un problème qui consiste à maximiser le nombre de cibles trouvées dans une certaine zone ayant une superficie connue et des caractéristiques qui lui sont propres (Présence d’obstacles, nombre de cibles, ..etc). La planification des actions des robots devient alors un défi faisant face à des contraintes liées aux caractéristiques de l’environnement.

Notre projet consiste à apporter des solutions à la recherche de cibles en adaptant des approches d’intelligence en essaim à cette thématique, communément appelées "méta-heuristiques". Ces approches n’ont pas encore été adaptées sur ce type de problème et qui sont :\textit{ BSO et Multi-BSO} s’inspirant de l’intelligence d’essaim d’abeilles, \textit{EHO et ESWSA} basées groupes d’éléphants qui s’inspirent de la mémoire individuelle des éléphants ainsi que l’intelligence des clans. Elles vont devoir trouver un nombre maximum de cibles en évitant les obstacles potentiellement présents dans l’environnement inconnu.
	\paragraph*{Mots clés:}
	Systèmes Multi-robots, Recherche de cibles, Évitement d'obstacles, Méta-heuristiques, Intelligence en essaim, BSO, EHO, ESWSA, Multi-BSO.
	
\end{poliabstract}

%\selectlanguage{english}
\begin{poliabstract}{Abstract}
The borderline between fiction and reality is getting thinner every day due to technological developments that have emerged over the years. Robots are no longer just part of science fiction. They have become part of our daily lives, offering new abilities to man for his everyday life.
Multi-robot systems have conquered the world of research due to the cooperation that makes robots intelligent as a group. Several studies are undertaken toward improving them by making them more robust and efficient.
Target search is one of their most recently exploited fields of application. It consists in maximizing the number of targets found in a certain area that we call environment. The planning of robot actions becomes a challenge facing constraints related to the characteristics of the environment.
Our project consists in providing solutions for Target search problem by adapting swarm intelligence approaches, commonly called "metaheuristics" which have not yet been adapted to this problem. The first two approaches, BSO and Multi-BSO are inspired by the intelligence of a swarm of bees, while the two last ones EHO and ESWSA are elephant group based that draws inspiration from the individual memory of elephants as well as the intelligence of clans. This metaheuristics will aim to find a maximum number of targets by avoiding potentially present obstacles on the unknown environment.
	\paragraph{Keywords:} 	Multi-robots system, Target search, Obstacles avoidance, Metaheuristics, Swarm intelligence, BSO,EHO , ESWSA, Multi-BSO.
\end{poliabstract}

\let\cleardoublepage\clearpage
